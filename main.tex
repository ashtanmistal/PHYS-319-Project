\documentclass[10pt]{article}
\usepackage{amsmath}
\usepackage{amssymb}
\usepackage{amsthm  }
\usepackage{fancyhdr}
\usepackage[margin=0.5in]{geometry}
\usepackage{graphicx}
\usepackage[utf8]{inputenc}
\usepackage{listings}
\usepackage{pdfpages}
\usepackage{standalone}
\usepackage{titling}
\usepackage{braket}
\usepackage{color}
\usepackage{hyperref}
\usepackage{wrapfig}


\definecolor{dkgreen}{rgb}{0,0.6,0}
\definecolor{gray}{rgb}{0.5,0.5,0.5}
\definecolor{mauve}{rgb}{0.58,0,0.82}
\lstset{
  language=C,
  aboveskip=3mm,
  belowskip=3mm,
  showstringspaces=false,
  columns=flexible,
  basicstyle={\small\ttfamily},
  numbers=none,
  numberstyle=\tiny\color{gray},
  keywordstyle=\color{blue},
  commentstyle=\color{dkgreen},
  stringstyle=\color{mauve},
  breaklines=true,
  breakatwhitespace=true,
  tabsize=4
}

\newcommand{\NN}{\mathbb{N}} % Naturals
\newcommand{\ZZ}{\mathbb{Z}} % Integers
\newcommand{\QQ}{\mathbb{Q}} % Rationals
\newcommand{\RR}{\mathbb{R}} % Reals
\newcommand{\CC}{\mathbb{C}} % Imaginaries
\newcommand{\HH}{\mathbb{H}} % Quaternions
\newcommand{\FF}{\mathbb{F}} % Field

\newcommand{\ud}{\,\mathrm{d}} % Single-var differential (use like \partial)

\newcommand{\CoulombConstant}{\frac{1}{4\pi\epsilon_0}}

\DeclareMathOperator{\erf}{erf} % Error Function
\DeclareMathOperator{\erfc}{erfc} % Complementary Error Function
\DeclareMathOperator{\erfi}{erfi} % Imaginary Error Function
\DeclareMathOperator{\row}{row} % Matrix Row
\DeclareMathOperator{\col}{col} % Matrix Column
\DeclareMathOperator{\trace}{tr} % Matrix Trace
\DeclareMathOperator{\proj}{proj} % Vector Projection

\title{PHYS 319 Final Project: PWM-Based Synthesizer}
\author{Ashtan Mistal}
\date{April 2022}

\begin{document}

\maketitle

\break

\tableofcontents{}

\break

\section{Abstract}

% [abstract here]

\section{Introduction}

% [introduction here]
Exploration into the possibility of creating a user-controlled, MIDI\footnote{musical instrument digital interface}-compatible fully featured synthesizer has held a personal interest shortly after I became interested in music production itself. Motivation for this project to encompass this personal interest stemmed from first using pulse width modulation to change the brightness of a bulb during an earlier lab. Since my more recent interest in audio equipment itself, I was motivated to do an audio-related project. The purpose of this project, as a result, was to help me gain a further understanding into the inner workings of oscillators and envelope generators, gain an appreciation for the MIDI standard, and of course to have a working, hand-made synthesizer for future music production related use. All three areas of knowledge mentioned above would aid in future audio-related projects, both in software as well as in hardware, and help inspire for future related projects. This project builds off of my previous related experience in music production, which provided me with a background on audio synthesis and processing. 

\section{Theory}

\subsection{Digital - Analog Converter}

\subsection{Analog - Digital Converter}

\subsection{Low Pass Filter}

\subsection{MIDI Standard}

\subsection{PWM-Based Synthesis}

\subsection{Envelope Generation}


\section{Apparatus}

\subsection{Apparatus Details}

The final rendition of this project is relatively simple in terms of electronics. We begin by sending a signal of the same voltage to 10 different buttons, which act as the user control for the frequency of the wave emitted from the PWM. These buttons then go into a 10-bit DAC, which was done using a resistor ladder based on the R-2R architecture, where the resultant signal is read through a 10-bit ADC (thus allowing for the same level of precision for the DAC as is available for the ADC) that is built into the MSP430G2553 microprocessor. % expand more

% include circuit diagram

\subsection{How the project is used}

The project is intended to be used similarly to a piano, with the 10 buttons available corresponding to various keys on a piano, ranging from A4 to C6, excluding sharps and flats in the note range. 

\subsection{Previous Apparatus Renditions}

As the initial intention of this project was to include MIDI compatibility, it is important to also include circuit diagrams for this, given it offered different methods of user interaction. The rendition of the project that allowed for MIDI input was the 3 oscillator version (which, as will be discussed in the Results section, did not work properly due to hardware limitations). Thus, the apparatus diagram below will reflect this 


\section{Results}


% TODO: 
% - Measure input on a FFT on Studio One (a spectral filter may be more accurate)
% - Compare to proper square wave
% - accuracy of the square wave frequency using a tuner



\section{Discussion}




\section{Conclusions}



%include references here

\section{Appendix}

\subsection{Appendix A: Table of Notes to Frequency Values}

\begin{table}[h]
    \centering
    \begin{tabular}{c|c}
         &  \\
         & 
    \end{tabular}
    \caption{Table of Notes to Frequency Values}
    \label{tab:pianonotestofrequency}
\end{table}

\end{document}
