% Description of the project including diagrams of electrical and mechanical aspects. Depending on the complexity of the circuitry, a separate block diagram that shows functional blocks might be beneficial in addition to complete electrical schematics. The text should provide details on how the apparatus works. This section should also include a description of how your project is used.

\subsection{PWM Based Synthesis Apparatus}\label{subsec:pwm-based-synthesis-apparatus}

The actual PWM was generated on the MSP430G2553 microcontroller.
There was no need to send the signal to a DAC, as the PWM signal was directly sent to the AUX output.
This AUX output was then connected to a speaker.
The goal of having an AUX output was to be able to analyze the signal on a computer, instead of having to use a microphone to record the signal which would lead to a less accurate representation of the signal.
This also allowed for the synthesizer to be connected to a more powerful speaker.

The actual circuit diagram for the output is very simple, and will be included in Section~\ref{subsec:apparatus-details}.


\subsection{Apparatus Details}\label{subsec:apparatus-details}

The final rendition of this project is relatively simple in terms of electronics.
We begin by sending a signal of the same voltage to 10 different buttons, which act as the user control for the frequency of the wave emitted from the PWM\@.
These buttons then go into a 10-bit DAC, which was done using a resistor ladder based on the R-2R architecture, where the resultant signal is read through a 10-bit ADC (thus allowing for the same level of precision for the DAC as is available for the ADC) that is built into the MSP430G2553 microprocessor. % expand more

% include circuit diagram

\subsection{How the project is used}\label{subsec:how-the-project-is-used}

The project is intended to be used similarly to a piano, with the 10 buttons available corresponding to various keys on a piano, ranging from A4 to C6, excluding sharps and flats in the note range.
When these buttons are pressed, the corresponding amplitude of the input signal is changed by sending the bits to a 10 bit DAC\@.
This signal is decoded by the ADC, which changes the period of the PWM signal accordingly by modifying a global variable.
The duty cycle is then set to always be 50\%, by bit shifting the value of the global variable to the right by 1 bit.
Finally, the signal is outputted, and heard by the user. 

\subsection{Previous Apparatus Renditions}\label{subsec:previous-apparatus-renditions}

As the initial intention of this project was to include MIDI compatibility, it is important to also include circuit diagrams for this, given it offered different methods of user interaction.
The rendition of the project that allowed for MIDI input was the 3 oscillator version (which, as will be discussed in the Results section, did not work properly due to hardware limitations).
Thus, the apparatus diagram below will reflect this