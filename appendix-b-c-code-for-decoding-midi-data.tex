Please note that functions regarding the actual waveform generation were ignored here for space and organization purposes. 

\begin{lstlisting}[label={lst:lstlisting}]
//
// Created by Ashtan Mistal on 2022-03-24.
//

#include "msp430.h"
#include "msp430g2553.h"

unsigned long frequency;
unsigned long amplitude;
unsigned long noteon;
unsigned long square_wave_volume;
unsigned long noise_wave_volume;
unsigned long triangle_wave_volume;
unsigned long sawtooth_wave_volume;
unsigned long attack_rate;
unsigned long decay_rate;
unsigned long sustain_level;
unsigned long release_rate;

void send_output(int input) {
    P1OUT = input;
}

// take in MIDI note and convert to frequency
// midi note is a number between 0 and 127
int midi_to_frequency(int midi_note) {
    // this is a slow function because it declares a variable every time it is called
    // will be better to declare a global variable and use that instead
    int midi_to_frequency_table[128] = {
        8, 8, 9, 9, 10, 10, 11, 12, 12, 13, 14, 15, 16, 17, 18, 19, 20, 21, 23, 24, 25, 27, 29, 30, 32, 34, 36, 38, 41, 43, 46, 48, 51, 55, 58, 61, 65, 69, 73, 77, 82, 87, 92, 97, 103, 110, 116, 123, 130, 138, 146, 155, 164, 174, 184, 195, 207, 220, 233, 246, 261, 277, 293, 311, 329, 349, 369, 391, 415, 440, 466, 493, 523, 554, 587, 622, 659, 698, 739, 783, 830, 880, 932, 987, 1046, 1108, 1174, 1244, 1318, 1396, 1479, 1567, 1661, 1760, 1864, 1975, 2093, 2217, 2349, 2489, 2637, 2793, 2959, 3135, 3322, 3520, 3729, 3951, 4186, 4434, 4698, 4978, 5274, 5587, 5919, 6271, 6644, 7040, 7458, 7902, 8372, 8869, 9397, 9956, 10548, 11175, 11840, 12544};
    return midi_to_frequency_table[midi_note];
}


// midi data is sent in as serial peripheral data on port P2.0
// convert the serial port input to an 8-bit integer
// this is the MIDI data
int serial_to_midi() {
    int i;
    int midi_data = 0;
    for (i = 0; i < 8; i++) {
        midi_data |= (P2IN & 0x01) << i; // assuming P2.0 is a 1 or 0
        // may have to delay cycles here based on baud rate -- or may have to remove for loop if baud rate cannot be changed and number of cycles is too high
    }
    return midi_data;
}


// decode the midi data to determine if it is a note on or a continuous controller change
// return if it is a note on or a continuous controller change
// if it is a note on, return 1
// if it is a continuous controller change, return 2
// else return -1
int decode_midi_data(int midi_data) {
    if ((midi_data & 0xF0) == 0x90) {
        return 1;
    } else if ((midi_data & 0xF0) == 0xB0) {
        return 2;
    } else if ((midi_data & 0xF0) == 0x80) { // note off
        return 3;
    } else {
        return -1;
    }
}

void handle_note_on(int midi_note, int velocity) {
    frequency = midi_to_frequency(midi_note);
    amplitude = velocity;
}

void handle_note_off() {
    noteon = 0;
    amplitude = 0; // set amplitude to 0 to turn off the note; remove this if we are using ADSR
    // amplitude will be handled by the release phase UNLESS we are removing the ADSR envelope

}

void handle_continuous_controller_change(int controller_number, int controller_value) {
// case numbers are arbitrary and based on what input MIDI controller we're using
    switch(controller_number) {
        case 1:
            square_wave_volume = controller_value; 
            break;
        case 2:
            noise_wave_volume = controller_value; 
            break;
        case 3:
            triangle_wave_volume = controller_value; 
            break;
        case 4:
            sawtooth_wave_volume = controller_value; 
            break;
        case 5:
            attack_rate = controller_value;
            break;
        case 6:
            decay_rate = controller_value;
            break;
        case 7:
            sustain_level = controller_value;
            break;
        case 8:
            release_rate = controller_value;
            break;
        default:
            break;
    }
}


/*
 * Steps to decode MIDI data:
 * 1. Read the first byte
 * 2. If the first byte is a note on, read the second and third byte
 * 3. If the first byte is a continuous controller change, read the second and third byte
 * 4. If the first byte is neither, return -1 and ignore the rest of the bytes
 */

// this is the interrupt service routine
void decoder() {
    int midi_data = serial_to_midi();
    int midi_data_type = decode_midi_data(midi_data);
    if (midi_data_type == 1) {
        int midi_note = serial_to_midi();
        int velocity = serial_to_midi();
        handle_note_on(midi_note, velocity);
    } else if (midi_data_type == 2) {
        int controller_number = serial_to_midi();
        int controller_value = serial_to_midi();
        handle_continuous_controller_change(controller_number, controller_value);
    } else if (midi_data_type == 3) {
        int midi_note = serial_to_midi();
        int velocity = serial_to_midi();
        handle_note_off();
    }
}




// enable an interrupt service routine based on the midi clock pin
void enable_interrupt() {
    P2IE |= 0x01; // enable interrupt on P2.0
    P2IES |= 0x01; // set interrupt to trigger on falling edge
    P2IFG &= ~0x01; // clear interrupt flag
    P2IE |= 0x01; // enable interrupt on P2.0
    _BIS_SR(GIE); // enable global interrupts
}

void __attribute__((interrupt(PORT2_VECTOR))) PORT2_ISR(void) {
    decoder(); // run ISR
    P2IFG &= ~0x01; // clear interrupt flag
}

int main(void) {
    P1DIR = 0b11111111; // set all pins on P1 to output
    P1OUT &= 0b00000000; // set all pins on P1 to 0
    // set up P2 input pins
    P2DIR = 0b00000000; // set all pins on P2 to input
    P2REN |= 0b11111110; // enable pull-up resistors on all pins on P2 except P2.0

    //// NOTE TO SELF that there is not in fact a clock pin on the MIDI cable.

    // initialize variables to some default values
    frequency = 2;
    amplitude = 255;
    noteon = 1; // can change to 0 if we do not want our synthesizer to be making noise when we turn it on
    square_wave_volume = 255;
    noise_wave_volume = 0;
    triangle_wave_volume = 255;
    sawtooth_wave_volume = 255;
    attack_rate = 2;
    decay_rate = 40;
    sustain_level = 127;
    release_rate = 127;



    // set up P2 interrupt
    enable_interrupt();
    unsigned long period = 1;

    while (1) {
        period = 1000000 / frequency; // period converted to microseconds
        unsigned long i;
        for (i = 0; i < period; i++) {
            unsigned long output = square_wave_volume * square_wave_iteration(period, i);
            output += noise_wave_volume * noise_wave_iteration(period, i)/4096;
            output += triangle_wave_volume * triangle_wave_iteration(period, i)/period;
            output += sawtooth_wave_volume * sawtooth_wave_iteration(period, i)/period;
            output = output / 4;
            output = output * amplitude / 128;
//            output = ADSR(output, timestep_counter);
//            serial_output(output);
            send_output((int) output);
        }
}
\end{lstlisting}