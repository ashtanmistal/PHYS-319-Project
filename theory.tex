% A presentation of the theoretical background necessary to understand your project. Not every project will need a theory section, but most will. Here you would describe the physics of how sensors you use work (eg how does accelerometer measure acceleration, or how does a capacitive touch sensor work), or how an electrical interface to a device works.

In the following subsections, some basics regarding audio synthesis is discussed, as well as basics regarding the conversion methods needed for analog synthesizers. These basics provide a sufficient enough understanding behind the components necessary for both the 3 oscillator, MIDI compatible synthesizer, as well as the PWM-based synthesizer. 

\subsection{Digital - Analog Converter}\label{subsec:digital---analog-converter}

The primary goal of a digital to analog converter is to convert data bits (strictly HIGH or LOW), and convert to an analog signal with voltage varying on the values of the data bits.
This can be done either by receiving all data pins at once, or sending the data pins individually along with a clock bit, which is then read by a segment in the integrated circuit that decodes the data.
Receiving all data pins at once does not require the use of an integrated circuit, and can be done using resistor ladders \cite{TERRELL1996337}. 
More details regarding resistor ladders will be shown in Section~\ref{subsec:apparatus-details}.
Nonetheless, the DAC allows us to input a certain number of bits, in the case of this project, 10 bits, and convert to an analog signal (thereby giving us 1024 different values from 0 volts up until the maximum voltage).
This is essentially "receiving what output voltage to emit in binary", so that 1111111111, all pins on, corresponds to the maximum voltage, and 0000000000 (all pins off) correspond to minimum voltage, and something like 1101100110 till be $\frac{870}{1023}$ of the maximum. 

\subsection{Analog - Digital Converter}\label{subsec:analog---digital-converter}

An analog to digital works in the exact opposite way as a digital to analog converter.
By taking in an input voltage (as a proportion of the maximum voltage), an analog to digital converter will set some data value to be equal to that input voltage as a binary proportion to the maximum (or reference) voltage.
In the case of the microprocessor, this reference will be 3.3 volts.
The usage of this in the project is to be able to perform certain actions based on the value of this input pin, i.e. control the frequency based on what notes are being pressed. 

\subsection{Low Pass Filter}\label{subsec:low-pass-filter}

The theory behind an analog low pass filter is relatively simple;
the signal is sent through an RC circuit (see figure~\ref{fig:LPF}), which, due to the charging and discharging properties of a capacitor, allows us to block frequencies past a certain cutoff frequency.
This cutoff frequency is given by the following formula:

\[
f_{cutoff}=\frac{1}{2 \pi R C}
\]

So, in order to reduce unnecessary signal noise, we can apply a low pass filter to block frequencies past the human range of hearing, as well as attenuate higher pitched distortion coming from the circuit. 

In the specific case of this project, a 9.55 nF capacitor was made available.
Using the formula mentioned above, we can set the cutoff frequency to 20 kHz, therefore using a resistor of 800 Ohms. 

\begin{figure}[H]
    \centering
    \includegraphics[width = 0.4 \textwidth]{lowpassfilter}
    \caption{A basic circuit diagram of a low pass filter. \cite{keim_2019}} % cite here
    % https://www.allaboutcircuits.com/technical-articles/low-pass-filter-tutorial-basics-passive-RC-filter/
    \label{fig:LPF}
\end{figure}

We can graph what the resultant signal graph will be post-LPF by using the following formula.
This formula is derived from the formula for a voltage divider circuit.



\[
V_{out}=V_{in}\left(\frac{X_{C}}{\sqrt{R_{1}^{2}+X_{C}^{2}}}\right), \quad X_{C}=\frac{1}{2 \pi f C}
\]

Hence, assuming $V_{in}$ is kept constant at 3.3 volts, we can view the output voltage as a function of the frequency, with $C = 9.55*10^{-9}$ F, and $R_1 = 800 \Omega$. 

\begin{figure}[H]
    \centering
    \includegraphics[width = 0.5 \textwidth]{lpfgraph}
    \caption{Plot of voltage versus frequency (linear scale) for an RC low pass filter. }
    \label{fig:lpfgraph}
\end{figure}

\subsection{MIDI Standard}\label{subsec:midi-standard-theory}
% https://www.instructables.com/Send-and-Receive-MIDI-with-Arduino/

Next, we delve further into details regarding the MIDI standard.
Note that this is strictly regarding the standard, and details regarding receiving MIDI data on the microprocessor will be described in Section~\ref{subsec:previous-apparatus-renditions}.

Sending MIDI data works by sending a stream of bytes at a given baud rate\footnote{The rate at which information is transferred in a communication channel.}. \cite{phongchit_2016} %https://www.setra.com/blog/what-is-baud-rate-and-what-cable-length-is-required-1#:~:text=The%20baud%20rate%20is%20the,of%209600%20bits%20per%20second.
This data is then decoded and processed, and instrument-related actions are taken accordingly. 

MIDI data is sent in either 2 or 3 bytes of information, depending on what the first byte is.
The first byte is always a command byte: \cite{ghassaei_instructables_2017}

\[
\begin{array}{l|l}
10000001 & \text { note off } \\
10010001 & \text { note on } \\
10100001 & \text { aftertouch } \\
10110001 & \text { continuous controller } \\
11000001 & \text { patch change } \\
11010001 & \text { channel pressure } \\
11100001 & \text { pitch bend } \\
11110001 & \text { non-musical commands }
\end{array}
\]

The latter half-byte in the command is what channel the command is being sent on (in this case, they are all being sent on channel 1. 
With specific regards to this project, MIDI commands were just being sent through Channel 1 for simplicity's sake. 

If the command is a Note On, Note Off, or aftertouch, the first byte that is sent after the command is what note was pressed, and the second is the velocity value or aftertouch value, respectively. If the command is a controlled change or similar, the first byte that is sent after the command is what controller number was changed, and its new value. We are ignoring aftertouch, patch change, channel pressure, pitch bend, and non-musical commands in this project. 

Hence, we can read the first byte, quickly decode it, and read the second byte and third byte accordingly.
The C implementations are in Appendix~\ref{sec:appendix-b:-c-code-for-decoding-midi-data}.

When reading a note on / off command, the second byte corresponds to what note is being played.
There is a direct conversion from the MIDI note to the frequency, through the following formula:

% include formula to convert from midi to frequency
\[
f_{note}=440\times\frac{2^{(n-69)/12}}{1}
\]

where $n$ is the MIDI note number.

This frequency is then converted into the period of the note, and then converted into microseconds.

Because this formula is expensive to calculate, we used a lookup table to store the values of the frequencies.
This lookup table is in Appendix~\ref{sec:-table-of-notes-to-frequency-values}.

\subsection{PWM-Based Synthesis}\label{subsec:pwm-based-synthesis-theory}

PWM-based synthesis generates a square waveform from a given frequency, and a given amplitude.
The waveform generator is built into the microprocessor, and is controlled by the period and duty cycle of the waveform.

The theory behind this is relatively straightforward.
Below is a diagram of the PWM waveform.

\begin{figure}[H]
    \centering
    \includegraphics[width = 0.5 \textwidth]{pwmwaveform}
    \caption{PWM waveform \cite{pwm}} %\href{https://en.wikipedia.org/wiki/Pulse-width_modulation#/media/File:Duty_Cycle_Examples.png}}
    \label{fig:pwmwaveform}
\end{figure}

For this project, we used the PWM generator to generate a square waveform which was then outputted to an AUX output.

The PWM was used once it was determined that multi-oscillation was too computationally expensive (more discussed in Section~\ref{subsec:pwm-based-synthesis-apparatus}).


\subsection{ADSR Envelope Generation}\label{subsec:envelope-generation-theory}

Modern synthesizers do not just have a note ON / OFF transition when playing. An envelope is what allows one oscillator to be played like a pluck, to a pad, to  numerous other types of sounds. An ADSR envelope is controlled by 4 crucial parts:


\begin{enumerate}
    \item Attack phase. The attack phase controls how long it takes from the initial note press to when the amplitude is at its maximum. When the attack is short (like we'd want on a pluck), the maximum amplitude is reached very quickly, whereas when the attack is long (like we might want on a pad or string-type instrument), the maximum amplitude is reached much more slowly. 
    \item Decay Phase. Once the note reaches its maximum amplitude from the attack stage and the note is still on, the decay phase controls the time it takes for the note to decay from the attack phase to the sustain level (explained below). For a pluck-like instrument, we will want this decay time to be relatively short, for example, as this shapes the sound to only be heard for a short amount of time regardless of the time that the note is played for. 
    \item Sustain Phase. So long as the note is still on, this is the amplitude of the output (as a fraction of the input amplitude, so a note with a lower velocity will have a sustained amplitude of the same fraction, but not the same volume, as a note with a higher velocity). 
    \item Release Phase. As soon as the note is released (regardless of what phase it is in), the envelope will go into the Release phase, where the rate at which the output level decays from its initial volume before the release phase to its final level (0 volume, or -$\infty$ dB) is controlled. For a slow release, the release would be long (likely useful for pads or other slow-decaying instruments). 
\end{enumerate}

Due to debugging other issues, the envelope generation for this project did not work as anticipated (and was a low priority issue, compared to getting the waveform generators to work). Nonetheless, the code for the envelope generator in progress is shared in Appendix \ref{sec:appendix-c:-adsr-envelope-code}. 